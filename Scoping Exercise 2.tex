\documentclass[a4paper,12pt]{article}
\usepackage[english]{babel}
\usepackage[utf8]{inputenc}

%
% For alternative styles, see the biblatex manual:
% http://mirrors.ctan.org/macros/latex/contrib/biblatex/doc/biblatex.pdf
%
% The 'verbose' family of styles produces full citations in footnotes, 
% with and a variety of options for ibidem abbreviations.
%
\usepackage{csquotes}
\usepackage[style=verbose-ibid,backend=bibtex]{biblatex}
\bibliography{sample}

\usepackage{lipsum} % for dummy text

\title{Scoping Exercise 2: Computational Analysis}

\author{John Hundley - 45072973}

\date{\today}

\begin{document}
\maketitle

\section{Goal}

A significant problem I have encountered in my discipline of philosophy is the inability to search multiple documents for key terms/themes at once. Instead of shifting through documents one by one and identifying relevant areas, I wish to perform a single search of ‘theme X’. The result would be to have the search return with the location of said theme in the selected documents. The search would ideally also return highlighting the connection between selected texts. The search should highlight where the argument is similar, different, expanded, or negated across different texts. For example, if I wanted to find where the argument of ‘transcendental deduction’ is introduced in Kant and followed upon by Hegel, I would need the search to firstly, show the corresponding areas, and, secondly, where they might differ. 

\section{Decomposition (Pains):}

The process is usually as follows. Establish argument/theme/problem to be discussed. Identify relevant terms to find sources. Search on a platform such as the Macquarie University Library. Identify and access required sources with reference to original intention. Select texts and documents that are relevant. Download documents. Open documents. Read documents. After reading decide on its relevance. Store any useful documents and discard any unnecessary. Then use the bibliographies of relevant documents to find other sources. Repeat process. Once a set number of documents has been procured, identify main arguments and establish relevance to the problem. 

\section{Decomposition (Gains):}

Ideally, the new process would be as follows. Establish argument/theme/problem to be discussed. Identify relevant terms to find sources. Search on a platform such as the Macquarie University Library. Identify and access required sources with reference to original intention. Download documents. Perform a search in the program/search engine that is relevant to the sources. Program will identify key areas whether by term, theme, or argument across the documents. Additionally, search terms could be customized according to categories such as format, date, location, and language. 

\section{Pattern Recognition:}

The main pattern is the searching, accessing, and reading of key documents. This process could be accelerated with a program that functions such as that described above. 

\section{Algorithm Design:}

1.	Identify thesis or argument. 
2.	Identify relevant sources. 
3.	Search necessary databases.
4.	Access and download required sources. 
5.	Perform search in program/search engine. 
6.	Observe links between texts. Note similarities and differences. Identify if further texts are required. Read documents and write response. 


\end{document}